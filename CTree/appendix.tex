
\section*{Appendix A}

An equivalent but computational simpler formulation of the linear statistic
for case weights greater than one can be written as follows. Let $\a = (a_1,
\dots, a_{\ws})$, $a_l \in \{1, \dots, n\}, l = 1, \dots, \ws$, denote the
vector of observation indices, with index $i$ occuring $w_i$ times. 
Instead of recycling the $i$th observation $w_i$ times it is sufficient to
implement the index vector $\a$ into the computation of the test statistic
and its expectation and covariance. For one
permutation $\sigma$ of $\{1, \dots, \ws\}$, the linear statistic
(\ref{linstat}) may be written as 
\begin{eqnarray*}
\T_j(\LS, \w) = \vec \left(\sum_{k=1}^{\ws} g_j(X_{ja_k})
           h(\Y_{\sigma(\a)_k}, (\Y_1, \dots, \Y_n))^\top \right) \in \R^{p_jq}
\end{eqnarray*}
now taking case weights greater zero into account.

\section*{Appendix B}

The results shown in Section~\ref{illustrations} are, up to some labelling, 
reproducible using the following \textsf{R} code:

\renewcommand{\baselinestretch}{1}

\begin{Sinput}
library("party")

data("GlaucomaM", package = "ipred")
plot(ctree(Class ~ ., data = GlaucomaM))

data("GBSG2", package = "ipred")  
plot(ctree(Surv(time, cens) ~ ., data = GBSG2))

data("mammoexp", package = "party")
plot(ctree(ME ~ ., data = mammoexp))
\end{Sinput}
